\partabstractfp{主要用于临界区的访问atmoic,spin\_lock, spin\_lock\_save}
\partabstractrp{\hei{锁}}
\partabstractlettrine{锁}{是Linux中重要的一部分} % the first word of the abstract

\part{Linux内核锁}

\chapter{Linux锁的种类及用法}

\section{语义整体:atmoic的用法}
\subsection{用atmoic定义变量的的原因}

\section{锁:spin\_lock关调度}

\section{中断:irq\_disable关中断}

\section{信号量:mutex调度睡眠}

\section{万能Linux锁用法}
锁的使用,主要是会了防止线程与线程,线程与中断,中断与中断之间的竞争,保护临界资源。如何正确地使用锁可以达到保护临界资源的目的呢。
我们先总结归纳下竞争的类型:
\begin{itemize}
  \item 核内竞争:\begin{enumerate}
               \item 核内线程与核内线程
               \item 核内线程与核内中断               
             \end{enumerate}
  \item 核间竞争:\begin{enumerate}
               \item 核间线程与核间线程
               \item 核间线间与核间中断
               \item 核间中断与核间中断
             \end{enumerate}
\end{itemize}
Linux锁的种类有如下几种:
\begin{itemize}
  \item 关闭抢占
  \item 关闭中断
  \item 不关闭抢占与中断,用信号量调度
\end{itemize}

\section{死锁检测}

\chapter{Linux锁答疑}
