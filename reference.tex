

\partabstractfp{}
\partabstractrp{参考的文件以PDF附件的形式,可以双击链接打开或保存,需选择支持PDF附件的PDF阅读器,建议使用adobe的阅读器打开附件}
\partabstractlettrine{参}{考这章列举了用到的相关资料源地址} % the first word of the abstract
\part{参考资料}
\chapter{参考文献}
\section{宋宝华相关网站资源}
\begin{enumerate}
  \item \href{https://edu.csdn.net/course/detail/5995}{CSDN视频课程 打通Linux脉络系列:进程、线程和调度}
  \item linux公众号:\textbf{Linux阅码场}
\end{enumerate}

\section{相关文章网址}
\begin{enumerate}
  \item \href{https://blog.csdn.net/feglass/article/details/46403501}{Android提权漏洞分析——rageagainstthecage}
  \item \href{https://github.com/21cnbao/training/blob/master/kernel/drivers/globalfifo/ch12/globalfifo.c}{globalfifo.c github源码地址}
  \item 
\end{enumerate}

\chapter{相关附件}
\section{pdf课件}
\ifattachpdffile
4天课程的PPT讲义,请用支持PDF附件的阅读器打开本文档,双击打开附件或在附件中另存为处理。
\else
4天课程的PPT讲义,链接在腾讯微云上。
\fi
\begin{enumerate}
  \item \videoattach{process_schedule_lesson1.pdf}{第1天进程讲义PDF}{https://share.weiyun.com/5PZzSJk}
  \item \videoattach{process_schedule_lesson2.pdf}{第2天进程讲义PDF}{https://share.weiyun.com/5X3kLcZ}
  \item \videoattach{process_schedule_lesson3.pdf}{第3天进程讲义PDF}{https://share.weiyun.com/5vuY8Jw}
  \item \videoattach{process_schedule_lesson4.pdf}{第4天进程讲义PDF}{https://share.weiyun.com/5CiC3Ew}
\end{enumerate}



\section{视频文件}
\ifattachpdffile
4天课程的所有的视频文件,请用支持PDF附件的阅读器打开本文档,双击打开附件或在附件中另存为处理。
\fi
\subsection{第一天视频文件}
\begin{enumerate}
  \item \videoattach{1- child-waited-by-parent.avi}{父进程获取子进程死亡原因.avi}{https://share.weiyun.com/5H21I1y}
  \item \videoattach{2- zombie.avi}{子进程变成僵尸.avi}{https://share.weiyun.com/5kkIww6}
  \item \videoattach{3- stop-status.avi}{进程stop状态.avi}{https://share.weiyun.com/5cqmSla}
  \item \videoattach{4- fork-printf-hello.avi}{用fork进程打印hello.avi}{https://share.weiyun.com/55sae7R}
  \item \videoattach{5- fork-ret.avi}{fork的返回值.avi}{https://share.weiyun.com/5kcl9vM}
\end{enumerate}




\subsection{第二天视频文件}
\begin{enumerate}
  \item \videoattach{1- cow.avi}{COW写时复制.avi}{https://share.weiyun.com/5UdNFye}
  \item \videoattach{2- vfork.avi}{vfork与COW区别.avi}{https://share.weiyun.com/5G2jlaC}
  \item \videoattach{3- thread.avi}{thread演示.avi}{https://share.weiyun.com/5EsHhl5}
  \item \videoattach{4- pid-tgid-pthread-self}{pid与tgid.avi}{https://share.weiyun.com/5wYury8}
  \item \videoattach{5- orphan.avi}{进程托孤.avi}{https://share.weiyun.com/57CLGS4}
  \item \videoattach{6- sleep-waitqueue.avi}{sleep和等待队列.avi}{https://share.weiyun.com/53SMfMt}
  \item \videoattach{7- idle-process.avi}{idle进程.avi}{https://share.weiyun.com/5et9Oz8}
\end{enumerate}

\subsection{第三天视频文件}
\begin{enumerate}
  \item \videoattach{0 - IO-CPU.avi}{CPU消耗和IO消耗.avi}{https://share.weiyun.com/5FwLvlt}
  \item \videoattach{1- scheduler.avi}{linux调度算法.avi}{https://share.weiyun.com/5yQ78xI}
  \item \videoattach{2- renice.avi}{renice设置进程.avi}{https://share.weiyun.com/5kRR0lo}
  \item \videoattach{3- nice.avi}{nice设置进程.avi}{https://share.weiyun.com/50Ij0ba}
  \item \videoattach{4- chrt.avi}{chrt设置进程.avi}{https://share.weiyun.com/5c3Upbl}
  \item \videoattach{5- rt-runtime-us.avi}{rt进程runtime熔断时间设置.avi}{https://share.weiyun.com/5pXmuXc}
\end{enumerate}


\subsection{第四天视频文件}
\begin{enumerate}
  \item \videoattach{1- load-balance.avi}{负载均衡举例.avi}{https://share.weiyun.com/5LywD1h}
  \item \videoattach{2- taskset.avi}{进程负载均衡taskset处理.avi}{https://share.weiyun.com/5fJGDQm}
  \item \videoattach{3- interrupt-affinity.avi}{中断负载均衡处理.avi}{https://share.weiyun.com/5gpf62s}
  \item \videoattach{4- cgroups-shares.avi}{cgroup的-shares参数.avi}{https://share.weiyun.com/5VBv6AQ}
  \item \videoattach{5- cgroup-quota.avi}{cgroup的-quota参数.avi}{https://share.weiyun.com/5vY9Q4A}
  \item \videoattach{6- why-not-rt.avi}{Linux与硬实时系统.avi}{https://share.weiyun.com/5uPnnwZ}
\end{enumerate}

%%% Local Variables:
%%% TeX-master: "main"
%%% End:
